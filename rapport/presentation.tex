\documentclass{beamer}

\usepackage[utf8]{inputenc} 
\usepackage[T1]{fontenc}
\usepackage{lmodern}
\usepackage{graphicx}
\usepackage[french]{babel}

\usetheme{Warsaw}

\title{Projet de système d'exploitation}
\subtitle{Soutenance}
\author{Tony SANCHEZ, Ludovic HOFER}
\institute{ENSERIB-MATMECA}
 
\begin{document}
 
\maketitle

\tableofcontents
 
\begin{frame}
\section{Introduction}
\frametitle{Introduction}

\begin{itemize}
\item Implémentation minimale de l'interface \verb!pthread!
\item Ajout du support de fonctionnalités avancées
\item Validation du fonctionnement par des tests
\item Comparaison des performances
\end{itemize}
 
\end{frame}

\section{Principes de fonctionnement général}
\subsection{Représentation des structures}

\begin{frame}
\frametitle{Structure de thread}
 
Contenu du transparent.
 
\end{frame}

\subsection{Changement de contexte}

\begin{frame}
  \frametitle{Changement de contexte}
  Contenu du transparent.

\end{frame}

\subsection{Libération des ressources}

\begin{frame}[containsverbatim]
  \frametitle{Libération des ressources}
  \begin{itemize}
  \item Destruction des threads
    \begin{itemize}
      \item Lorsque le résultat est récupéré par \verb!thread_join!
    \end{itemize}
  \item Mémoire des piles
    \begin{itemize}
    \item En même temps que les threads $\rightarrow $ Mauvaise idée!
    \item À libérer dès l'appel à \verb!thread_exit(...)!
    \end{itemize}
  \item Empreinte mémoire de la bibliothèque
    $\rightarrow $ \verb!atexit(...)!
    \begin{itemize}
      \item Permet de valider par des tests l'absence de fuites mémoires
        simplement.
      \item Important pour une bibliothèque
      \item Problèmes internes aux GList $\rightarrow $ Abandon de ces tests
    \end{itemize}
  \end{itemize}
\end{frame}

\subsection{Ordonnancement}

\begin{frame}
  \frametitle{Ordonnancement}
  \begin{itemize}
    \item Version mono-core
    \item Version multi-core
  \end{itemize}
\end{frame}

\subsection{Mutex}

\begin{frame}[containsverbatim]
  \frametitle{Mutex}
  \begin{itemize}
    \item Utilisation des \verb!pthread_mutex!
    \item \verb!thread_mutex_lock!
      \begin{itemize}
        \item boucle sur \verb!pthread_mutex_trylock!
          \begin{description}
          \item[Succès :] sort de la boucle
          \item[Échec :]  \verb!thread_yield!
          \end{description}
        \item Utilisation de \verb!try_lock! nécessaire pour le
          multi\_coeur sans préemption.
      \end{itemize}
      \item Idée
        \begin{itemize}
        \item Associer une file de threads en attente à chaque verrou
        \item Changer le statut du thread
        \item Avantage : évite l'attente active et des changements de
          contexte inutiles
        \end{itemize}
  \end{itemize}
\end{frame}

\subsection{Support multi-coeur}

\begin{frame}[containsverbatim]
  \frametitle{Support multi-coeur}
  \begin{itemize}
    \item Utilisation de \verb!k! threads noyaux
      \begin{itemize}
      \item Défini à la compilation
      \item Présence des \verb!k! threads noyaux, même s'il y a un unique
        thread actif
      \end{itemize}
    \item Si \verb!k! est plus grand que \verb!nb_threads_actifs!
      \begin{itemize}
        \item Les threads noyaux en trop sont en attente active sur un
          \verb!sleep!
        \item Après chaque \verb!sleep!, les threads parcourent la liste de
          thread
      \end{itemize}
    \item Utilisation de \verb!pthread_mutex! afin de synchroniser les
      threads noyaux
    \item Implémentation difficile et peu efficace à cause des listes
      \begin{itemize}
      \item Nécessité de conserver l'indice de chaque thread.
      \item Parcours de toute la liste même si aucun thread n'est disponible
      \item Lors d'un \verb!thread-join!, mise à jour de tous les index
      \end{itemize}
  \end{itemize}
\end{frame}

\section{Tests}

\begin{frame}[containsverbatim]
  \frametitle{Tests}
  \begin{itemize}
    \item Script permettant de tester tous les tests avec affichage des
      résultats
    \item Avant les GList : validation des tests par \verb!valgrind!
  \end{itemize}
\end{frame}

\section{Performances}

\begin{frame}
  \frametitle{Performances}
\end{frame}

\section{Perspectives}

\begin{frame}
  \frametitle{Perspectives}
\end{frame}

\section{Conclusion}

\begin{frame}
  \frametitle{Conclusion}
\end{frame}


 
\end{document}
