\documentclass{beamer}

\usepackage[utf8]{inputenc} 
\usepackage[T1]{fontenc}
\usepackage{lmodern}
\usepackage{graphicx}
\usepackage[french]{babel}

\usetheme{Warsaw}

\title{Projet de système d'exploitation}
\subtitle{Soutenance}
\author{Tony SANCHEZ, Ludovic HOFER}
\institute{ENSERIB-MATMECA}
 
\begin{document}
 
\maketitle

\tableofcontents
 
\begin{frame}
\section{Introduction}
\frametitle{Introduction}
 
Contenu du transparent.
 
\end{frame}

\section{Principes de fonctionnement général}
\subsection{Représentation des structures}

\begin{frame}
\frametitle{Structure de thread}
 
Contenu du transparent.
 
\end{frame}

\subsection{Changement de contexte}

\begin{frame}
  \frametitle{Changement de contexte}
  Contenu du transparent.

\end{frame}

\subsection{Libération des ressources}

\begin{frame}
  \frametitle{Libération des ressources}
  \begin{itemize}
  \item Destruction des threads
  \item Mémoire des piles
%    \begin{itemize}
%    \item En même temps que les threads $\rightarrow $ Mauvaise idée!
%    \item À libérer dès l'appel à \verb!thread_exit(...)!
%    \end{itemize}
  \item Empreinte mémoire de la bibliothèque
    $\rightarrow $ \verb!atexit(...)!
  \end{itemize}
\end{frame}

\subsection{Ordonnancement}

\begin{frame}
  \frametitle{Ordonnancement}
  \begin{itemize}
    \item Version mono-core
    \item Version multi-core
  \end{itemize}
\end{frame}

\section{Tests}

\begin{frame}
  \frametitle{Tests}
\end{frame}

\section{Performances}

\begin{frame}
  \frametitle{Performances}
\end{frame}

\section{Perspectives}

\begin{frame}
  \frametitle{Perspectives}
\end{frame}

\section{Conclusion}

\begin{frame}
  \frametitle{Conclusion}
\end{frame}


 
\end{document}
