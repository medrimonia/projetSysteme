\documentclass{beamer}

\usepackage[utf8]{inputenc} 
\usepackage[T1]{fontenc}
\usepackage{lmodern}
\usepackage{graphicx}
\usepackage[french]{babel}

\usetheme{Warsaw}

\title{Projet de système d'exploitation}
\subtitle{Soutenance}
\author{Tony SANCHEZ, Ludovic HOFER, Henri GARDON, Pierre-Alain POCQUET}
\institute{ENSEIRB-MATMECA}
 
\begin{document}
 
\maketitle

\tableofcontents
 
\begin{frame}
\section{Introduction}
\frametitle{Introduction}

\begin{itemize}
\item Implémentation minimale de l'interface \verb!pthread!
\item Ajout du support de fonctionnalités avancées
\item Validation du fonctionnement par des tests
\item Comparaison des performances
\end{itemize}
 
\end{frame}

\section{Principes de fonctionnement général}
\subsection{Représentation des structures}

\begin{frame}
\frametitle{Structure de thread}
 
\begin{description}
	\item[context] Le contexte d'exécution du thread
	\item[stack\_id] Un attribut nécessaire pour valgrind
	\item[status] Le statut du thread
	\item[next] Le thread a exécuter à la suite d'un join
	\item[freeNeeded] Empêche de libérer les ressources du thread de base
	\item[retval] La valeur de retour du thread
\end{description}
 
\end{frame}

\begin{frame}
\frametitle{Les threads}

\begin{itemize}
\item Première version naïve et utilisation de tableau
\item Passage à une implémentation sous forme de liste
\item Amélioration possible et passage à une ou des file(s)
	
\end{itemize}
 
\end{frame}


\subsection{Changement de contexte}

\begin{frame}
  \frametitle{Changement de contexte}
  Le changement de contexte s'effectue en deux étapes
  
  \begin{description}
  \item[Recherche du prochain] Ici on parcours notre(nos) ensemble(s) de thread, jusqu'à trouver un candidat.
  \item[Changement de contexte] Puis grâce à la fonction \verb!swapcontext()! on opère au passage dans ce nouveau contexte.
	
\end{description}

\end{frame}

\subsection{Libération des ressources}

\begin{frame}[containsverbatim]
  \frametitle{Libération des ressources}
  \begin{itemize}
  \item Destruction des threads
  \item Mémoire des piles
    \begin{itemize}
    \item En même temps que les threads $\rightarrow $ Mauvaise idée!
    \item À libérer dès l'appel à \verb!thread_exit(...)!
    \end{itemize}
  \item Empreinte mémoire de la bibliothèque
    $\rightarrow $ \verb!atexit(...)!
  \end{itemize}
\end{frame}

\subsection{Ordonnancement}

\begin{frame}
  \frametitle{Ordonnancement}
  \begin{itemize}
    \item Version mono-core : \\
    Ordonnancement au sein des fonctions de la bibliothèque.
    \item Version multi-core : \\
    Ordonnancement au sein des threads noyaux.
  \end{itemize}
\end{frame}

\subsection{Mutex}

\begin{frame}
  \frametitle{Mutex}
%  \begin{itemize}
%    \item Utilisation des \verb!pthread_mutex!
%    \item \verb!thread_mutex_lock! est équivalent à une boucle sur
%      \verb!pthread_mutex_try_lock!
%  \end{itemize}
\end{frame}

\subsection{Support multi-coeur}

\begin{frame}
  \frametitle{Support multi-coeur}
  \begin{itemize}
    \item Utilisation de $k$ threads noyaux
      \begin{itemize}
      \item Défini à la compilation
      \item Présence des $k$ threads noyaux, même s'il y a un unique thread
        actif
      \end{itemize}
    \item Si $k$ est plus grand que $nb\_threads\_actifs$
      \begin{itemize}
        \item Les threads noyaux en trop sont en attente active sur un sleep
        \item Après chaque sleep, les threads parcourent la liste de thread
      \end{itemize}
  \end{itemize}
\end{frame}

\section{Tests}

\begin{frame}
  \frametitle{Tests}
\end{frame}

\section{Performances}

\begin{frame}
  \frametitle{Performances}
\end{frame}

\section{Perspectives}

\begin{frame}
  \frametitle{Perspectives}
\end{frame}

\subsection{Générales}

\begin{frame}
	\frametitle{Générales}
	
	\begin{itemize}
		\item Révision de l'architecture
		\begin{itemize}
			\item Séparation modulaire
			\item Fonctions de haut niveau
		\end{itemize}
		\item Passages aux files
		\begin{itemize}
			\item Gestion des priorités
			\item Amélioration des performance
		\end{itemize}
	
	\end{itemize}
\end{frame}

\subsection{Préemption}

\begin{frame}
  \frametitle{Préemption}
  \begin{itemize}
    \item Utilisation d'un timer virtuel
      \begin{itemize}
      \item signal $ITIMER\_VIRTUAL$ toute les $TIMESLICE$ sec
      \item décroissant lors de l'exécution du processus
      \item appel à $thread\_yield$ pour changer de contexte
      \end{itemize}
    \item Désactiver préemption pour les fonctions critiques.
    \item Test non concluant
  \end{itemize}
\end{frame}

\section{Conclusion}

\begin{frame}
  \frametitle{Conclusion}
\end{frame}


 
\end{document}
