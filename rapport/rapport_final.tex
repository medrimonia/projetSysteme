\documentclass{article}

\usepackage[french]{babel}
\usepackage[utf8]{inputenc}
\usepackage[T1]{fontenc}
\usepackage{verbatim}
\usepackage{amsmath}
\usepackage{geometry}
\usepackage{calc}
\usepackage{color}

%\date{}
\author{Groupe :\\ \\Gardon Henri\\Ludovic Hofer\\Jérôme Lebot\\Pierre-Alain Pocquet\\Tony Sanchez}
\title{Projet système \\ Rapport intermédiaire}

\begin{document}

\maketitle


\section{Représentation des threads}

\subsection{Structure de données}

L'identification et la gestion des threads manipulés par l'ordonnanceur reposent sur
la structure thread, dont les champs sont décrits comme suit :

    \noindent\parindent11.10839pt\fbox{\parbox{\linewidth-2\fboxrule-2\fboxsep}{
       
       \textcolor{blue}{struct} thread\{ \\
    	\indent	\textcolor{blue}{ucontext\_t} context;\\
    	\indent	\textcolor{blue}{int} stack\_id;\\
    	\indent	\textcolor{blue}{int} status;\\
    	\indent	\textcolor{blue}{bool} freeNeeded;\\
    	\indent	\textcolor{blue}{void} * retval;\\
		\};
	
	}}
\\
\begin{itemize}
    \item context : conserve le status associé à ce thread.
    \item stack\_id : ce champ permet de renseigner Valgrind sur l'identifiant de la pile où le thread est stocké. 
    \item status : il s'agit du status du thread. Il existe deux cas :
    \begin{description}
        \item[STATUS\_TERMINATED]le thread a terminé son exécution, sa valeur de retour se trouve dans \textit{retval}
    \end{description}
    Les autres status du thread n'interviennent pas.
    \item freeNeeded : ce champ indique si le thread est libéré.****A PRECISER****
    \item retval : est la valeur de retour dans le cas où le thread aurait fini son exécution (status TERMINATED cf \textit{status}), NULL sinon.
\end{itemize}

\noindent\fbox{\parbox{\linewidth-2\fboxrule-2\fboxsep}{
    
    \textcolor{blue}{struct} mutex\{  \\
    \textcolor{blue}{pthread\_mutex\_t} mutex; \\
    \};

}}


\subsection{Variables globales}

Afin de simplifier l'implémentation des fonctions, les variables globales définies ci-dessous ont été mises en places.

\begin{itemize}

    \item GList * threads : ****A COMPLETER****\\

    \item sigset\_t preempt\_set; ****A COMPLETER****\\

    \item int current\_thread cet entier indique l'identifiant du thread en cours d'exécution\\

    \item int working\_threads cet entier indique le nombre de threads dont l'exécution n'est pas terminée.\\

    \item int nb\_threads cet entier indique le nombre de threads existants. \\

    \item int nb\_threads\_waiting\_join : cet entier indique le nombre de thread attendant la fin de leur exécution.\\

    \item int next\_thread\_create : cet entier indique l'identifiant à donner au prochain thread à créer.

\end{itemize}

\subsection{Fonctions intermédiaires}

    \subsubsection{add\_thread()}

        \noindent\fbox{\parbox{\linewidth-2\fboxrule-2\fboxsep}{\textcolor{blue}{struct} thread * add\_thread();}}   

 \subsubsection{next\_thread()}

        \noindent\fbox{\parbox{\linewidth-2\fboxrule-2\fboxsep}{\textcolor{blue}{struct} thread * next\_thread();}}  

\subsubsection{wrapper()}

        \noindent\fbox{\parbox{\linewidth-2\fboxrule-2\fboxsep}{\textcolor{blue}{void} wrapper(\textcolor{blue}{void} *(*func)(\textcolor{blue}{void}*), \textcolor{blue}{void} * funcarg)); }}  

\subsubsection{thread\_self()}

        \noindent\fbox{\parbox{\linewidth-2\fboxrule-2\fboxsep}{\textcolor{blue}{thread\_t} thread\_self();}}

\subsubsection{initialize\_thread\_handler()}

        \noindent\fbox{\parbox{\linewidth-2\fboxrule-2\fboxsep}{\textcolor{blue}{void} initialize\_thread\_handler();}}

\subsubsection{free\_thread()}

        \noindent\fbox{\parbox{\linewidth-2\fboxrule-2\fboxsep}{\textcolor{blue}{void} free\_thread(\textcolor{blue}{struct} thread * t);}}

\subsubsection{end\_thread\_handling()}

        \noindent\fbox{\parbox{\linewidth-2\fboxrule-2\fboxsep}{\textcolor{blue}{void} end\_thread\_handling();}}

\subsubsection{thread\_mutex\_init()}

        \noindent\fbox{\parbox{\linewidth-2\fboxrule-2\fboxsep}{\textcolor{blue}{mutex\_p} thread\_mutex\_init();}}

\subsubsection{thread\_mutex\_lock()}

        \noindent\fbox{\parbox{\linewidth-2\fboxrule-2\fboxsep}{\textcolor{blue}{void} thread\_mutex\_lock(\textcolor{blue}{mutex\_p} mutex);}}

\subsubsection{thread\_mutex\_unlock()}

        \noindent\fbox{\parbox{\linewidth-2\fboxrule-2\fboxsep}{\textcolor{blue}{void} thread\_mutex\_unlock(\textcolor{blue}{mutex\_p} mutex);}}

\subsubsection{thread\_mutex\_destroy()}

        \noindent\fbox{\parbox{\linewidth-2\fboxrule-2\fboxsep}{\textcolor{blue}{void} thread\_mutex\_destroy(textcolor{blue}{mutex\_p} mutex);}}



\section{Choix d'implémentation des fonctions}

\subsection{thread\_create()}
    


\end{document}
%\noindent\fbox{\parbox{\linewidth-2\fboxrule-2\fboxsep}{}}