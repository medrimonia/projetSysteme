\documentclass{article}

\usepackage[french]{babel}
\usepackage[utf8]{inputenc}
\usepackage[T1]{fontenc}
\usepackage{verbatim}
\usepackage{amsmath}
\usepackage{geometry}
\usepackage{calc}
\usepackage{color}

%\date{}
\author{Groupe :\\ \\Gardon Henri\\Ludovic Hofer\\Jérôme Lebot\\Pierre-Alain Pocquet\\Tony Sanchez}
\title{Projet système \\ Rapport intermédiaire}

\begin{document}

	\maketitle



	\section{Introduction}

		Au sein de ce projet nous nous attacherons à mettre en place une bibliothèque de gestion de threads. Cette dernière proposera une interface de programmation semblable à celle de pthread, à la différence près qu'un seul thread noyau sera utilisé.

	\section{Structures de données}

	\subsection{Représentation d'un thread}

	\subsection{Représentation des threads}

	\subsubsection{Première version et utilisation d'un tableau}

	\subsubsection{Amélioration et passage listes}

	\section{Fonctionnement de la bibliothèque}

	\subsection{ajout d'un thread}

	\subsection{Recherche du prochain thread à exécuter}

	\subsection{Spécificités d'implémentation}

	\section{ordonnancement}

	\subsection{Sans ordonnanceur}

	\subsection{Version multi-coeur et changements}

	\section{Tests et performance}

	\section{Ce qui est fait}

	\section{Perspective}

	\section{Conclusion}

\end{document}